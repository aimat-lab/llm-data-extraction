\addchap{Abstract}\label{chap:abstract}
\pagenumbering{arabic}

% How to write an abstract
%  One or two sentences providing a basic
%  introduction to the field, comprehensible to
%  a scientist in any discipline.
A majority of materials science knowledge is contained in unstructered text scattered throughout the body of scientific literature, out of reach for increasingly capable but data-starved \acrlong{ML} models that are being used more and more at every step of the materials creation process.
%  Two to three sentences of more detailed
%  background, comprehensible to scientists
%  in related disciplines.
Thus, the extraction of such information from unstructured scientific literature and conversion to machine-readable formats has become a challenge of \acrlong{NLP}.
Recently, \acrlongpl{LLM} have gained prominence for their general capabilities across \acrlong{NLP} tasks, including \acrlong{NER}.
%  One sentence clearly stating the general
%  problem being addressed by this particular
%  study.
%However, such information is often not accessible in a machine-readable format.
%  One sentence summarizing the main result
%  (with the words “here we show” or their
%  equivalent).
This work demonstrates that automated information extraction from materials science literature is possible with high accuracy using models of only 13 billion parameters and without fine-tuning.
%  Two or three sentences explaining what the
%  main result reveals in direct comparison to
%  what was thought to be the case previously,
%  or how the main result adds to previous
%  knowledge.
In fact, 13 billion parameter sized variants from both \model{llama} and \model{llama2} achieved an accuracy of 95 to 98\% for the extraction of temperature and time information from unstructured text.
Additionally, the 7 billion parameter sized \model{falcon} achieved an accuracy of 79\% on the extraction of solvent information from synthesis paragraphs on the creation of \acrlongpl{MOF}.
%Contrary to expectation, fine-tuning is harder than expected with few resources available.
%  One or two sentences to put the results
%  into a more general context.
The smaller size of these models, their open-access availability, and no necessity for additional fine-tuning enables the usage of most consumer hardware, making these capabilities far more accessible than previously expected.
%  Two or three sentences to provide a broader
%  perspective, readily comprehensible to a
%  scientist in any discipline.

% This means, that even for more sophisticated information extraction tasks fine-tuning might not be necessary, particularly with more generally capable models in the future.

\addchap{Zusammenfassung}
Deutsche Zusammenfassung hier. \todo{write zusammenfassung}


\cleardoublepage
% \pagenumbering{arabic}
% other page numbering styles:
% - arabic: use Arabic numerals (1, 2, 3, ...)
% - alph: use lowercase letters (a, b, c, ...)
% - Alph: use uppercase letters (A, B, C, ...)
% - roman: use lowercase roman numerals (i, ii, iii, ...)
% - Roman: use uppercase roman numerals (I, II, III, ...)



