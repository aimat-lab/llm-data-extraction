
% --------------------------------------
%           Acronyms
% --------------------------------------

\newacronym{MLP}{MLP}{Multi-Layer Perceptron}
\newacronym{LOC}{LOC}{Lines of Code}
\newacronym{ReLU}{ReLU}{Rectified Linier Unit}
\newacronym{SwiGLU}{SwiGLU}{Swish Gated Linear Unit \cite{shazeer_glu_2020}}
\newacronym{ML}{ML}{Machine Learning}
\newacronym{LM}{LM}{Language Model}
\newacronym{LLM}{LLM}{Large Language Model}
\newacronym{NER}{NER}{Named Entity Recognition}
\newacronym{ER}{ER}{Entity Recognition}
\newacronym{MOF}{MOF}{Metal Organic Framework}


\newacronym{TF-IDF}{TF-IDF}{Term Frequency - Inverse Document Frequency}
\newacronym{BM25}{BM25}{Okapi Best-Matching}
\newacronym{BERT}{BERT}{Bidirectional Encoder Representation from Transformers \cite{devlin_bert_2018}}
\newacronym{SOTA}{state-of-the-art}{State of the Art}
\newacronym{NLP}{NLP}{Natural Language Processing}
\newacronym{RoPE}{RoPE}{Rotary Positional Encoding \cite{su_roformer_2022}}
\newacronym{GQA}{GQA}{Grouped Query Attention \cite{ainslie_gqa_2023}}
\newacronym{GPT}{GPT}{Generative Pretrained Transformer}

\newacronym{transformers}{\texttt{transformers}}{\gls{hf} \texttt{transformers} \cite{huggingface_2023}}


% --------------------------------------
%           Terms
% --------------------------------------

\newglossaryentry{causal}{
    name=causal language model,
    description={A causal language model predicts the likelihood of the next token based on a sequence of tokens (input). By sampling one of the predicted tokens and appending it to the input, output can be generated autoregressively. This in contrast to e.g. a \gls{masked}.}
}

\newglossaryentry{masked}{
    name=masked language model,
    description={A masked language model predicts all masked (often missing) tokens in a sequence based on the context provided by the surrounding tokens. This in contrast to e.g. a \gls{causal}.}
}


\newglossaryentry{fscore}{
    name=F-Score,
    description={$ 2\cdot (precision \cdot recall) / (precision + recall)$, where \textit{precision} is a metric for how many retrieved items are relevant, and \textit{recall} is a metric for how many relevant items where retrieved.}
}

\newglossaryentry{pp}{
    name=percentage point,
    description={One percentage point denotes an absolute increase of 1\% in the underlying measure.
    }
}


% --------------------------------------
%           Models
% --------------------------------------



\newglossaryentry{GPT2}{
    name=GPT2,
    description={The second generation \textbf{G}enerative \textbf{P}retrained \textbf{T}ransformer \gls{LM} from \gls{OpenAI} \cite{radford_language_2019}.
    }
}

\newglossaryentry{GPT3}{
    name=GPT3,
    description={The third generation \textbf{G}enerative \textbf{P}retrained \textbf{T}ransformer \gls{LM} from \gls{OpenAI} \cite{brown_language_2020}.
    }
}

\newglossaryentry{ChatGPT}{
    name=ChatGPT,
    description={A chat interface for \gls{GPT3}.5 or \gls{GPT4} from \gls{OpenAI}, which makes it easier to interface with the model. \subref{instruct} provides more information on instruct-based models, chat-variants like ChatGPT are an extension of.
    }
}

\newglossaryentry{GPT4}{
    name=GPT4,
    description={The fourth generation \textbf{G}enerative \textbf{P}retrained \textbf{T}ransformer \gls{LM} from \gls{OpenAI} \cite{openai_gpt4_2023}. Currently their most capable model.
    }
}

\newglossaryentry{BLOOM}{
    name=BLOOM,
    description={\textbf{B}igScience \textbf{L}arge \textbf{O}pen-science \textbf{O}pen-access \textbf{M}ultilingual \gls{LM}, a 176 billion parameter open-source \gls{LLM} created through a cooperation between \gls{Google}, \gls{hf} and various smaller organisations \cite{workshop_bloom_2022}.
    }
}

\newglossaryentry{OPT}{
    name=OPT,
    description={Open Pretrained Transformer, a 175 billion parameter open-source \gls{LM} from \gls{meta} research \cite{zhang_opt_2022}.
    }
}

\newglossaryentry{PaLM}{
    name=PaLM,
    description={An extremely capable closed-access \gls{LLM} from \gls{Google} \cite{chowdhery_palm_2022}.
    }
}

\newglossaryentry{llama}{
    name=LLaMa,
    description={A \gls{LLM} from \gls{meta}. See \subref{llama} for details.
    }
}

\newglossaryentry{alpaca}{
    name=Stanford Alpaca,
    description={A \gls{LLM} based on \gls{llama}. See \subref{alpaca} for details.
    }
}

\newglossaryentry{vicuna}{
    name=Vicuna,
    description={One of the \glspl{LLM} used. Based on \model{llama}. See \subref{vicuna} for details.
    }
}

\newglossaryentry{llama2}{
    name=LLaMa 2,
    description={One of the \glspl{LLM} used. It is the successor of \model{llama}, also created by \gls{meta}. See \subref{llama2} for details.
    }
}

\newglossaryentry{falcon}{
    name=Falcon,
    description={One of the \glspl{LLM} used. Created by the \gls{tii}. See \subref{falcon} for details.
    }
}

% --------------------------------------
%           Organizations
% --------------------------------------

\newglossaryentry{meta}{
    name=Meta,
    description={Previously known as Facebook, Meta is a deep learning powerhose and regularly open-sources new \gls{SOTA} machine learning models.}
}

\newglossaryentry{OpenAI}{
    name=OpenAI,
    description={American AI company, trailblazer at the frontier of scaling deep learning architectures and corresponding algorithmic breakthroughs. Their currently most well-known models are the \gls{GPT} family of models, particularly \gls{GPT2}, \gls{GPT3} and \gls{GPT4}.}
}

\newglossaryentry{Google}{
    name=Google,
    description={Now called Alphabet, search engine giant and AI powerhouse, well known for training supersized machine learning models with unrealistic hardware requirements for just about anyone else.}
}

\newglossaryentry{microsoft}{
    name=Microsoft,
    description={Tech Giant, well-known for its operating system. Microsoft recently started intensive cooperation with \gls{OpenAI} through a \$10 Billion USD investment, and started integrating \gls{GPT4} and other models throughout their services.}
}


\newglossaryentry{hf}{
    name=HuggingFace,
    description={American deep learning ecosystem startup, having created the well established \texttt{transformers} framework which provides useful abstractions of most existing open-access Machine Learning models.}
}

\newglossaryentry{tii}{
    name=Technology Innovation Institute,
    text=TII,
    first={Technology Innovation Institute (TII)},
    description={Abu Dhabi-based machine learning research institute.}
}

\newglossaryentry{lmsys}{
    name=Large Model Systems Organization,
    text=LMSYS,
    first={Large Model Systems Organization (LMSYS)},
    description={The Large Model Systems Organization eponymously develops large models and systems that are open, accessible, and scalable, e.g. Vicuna.}
}

