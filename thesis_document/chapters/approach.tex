\chapter{Approach}\label{chap:approach}
\todo{approach chapter guideline: what did I do, why did I do it}
Here, we describe some of our approach to using \glspl{LLM}, decisions made, lessons learned, and more.

Specifically, we put little effort in prompt engineering, with the argument that it should have little impact for fine-tuned models, the main part of our work. As it turns out (See \secref{sft} for more details on training failures), this ended up not being the case.

See \subref{list} for a list of the models used, \subref{criteria} for selection criteria for the models, and the respective Subsection for more details on each model, which is als linked in their glossary entry.
\todo{rewrite when the rest becomes clearer}

All source code for this work can be found at \url{https://github.com/fkarg/mthesis}, and a tag will mark the state at the time of submission.

\section{Implementation}\label{sec:impl}
As partially described before in \secref{models}, it became appearent during literature research that it might be valuable to compare different models, and that additional models might become available in the near future.
The \acrlong{transformers} library is a well-established framework providing abstractions to load, manipulate and train any deep learning architecture in a standardized format.
Additionally, all open-access \glspl{LLM} are available directly through the \gls{hf} portal.

Most other libraries used are either straightforward (\texttt{torch, einops, accelerate, bitsandbytes} to get the models to run) or common ecosystem choices (e.g. \texttt{typer, rich} as cli interface).
Proportionally speaking, the custom dataloader(s) and the main module have the highest \gls{LOC} counts.

\section{Language Models Considered}\label{sec:models}
The following sections describe the various models considered for the benchmark, and the reasons for or against their inclusion.
Before that, \subref{criteria} depicts simple criteria for model selection.

See \secref{basics} for an overview and broad categorization of the models mentioned here.


\subsection{Criteria}\label{sub:criteria}
First, the model weights need to be available.
This is necessary to run the model in a self-hosted manner.
Practically, this is a constraint towards open-source or at least open-access models (their license needs to allow for academic research).
% \todo{reason for criteria: people want to fine-tune models on their own potentially sensitive data}

Second, at this point dozens of \glspl{LLM} have been trained, and weights made available.
One model was trained purely as a marketing pitch to sell computing systems \cite{dey_cerebrasgpt_2023}.
To vastly condense the number of models to consider, each base model ought to have demonstrated fundamental capability in other domains.
Additionally, due to the nature of the task, it is valuable to include both base models and instruct-based derivatives for comparison.

Third, all else being equal, a smaller model is preferred.
This is due to smaller models being less resource intensive to evaluate and fine-tune.
Smaller variants of larger models additionally enable faster and less resource intensive iteration during development.

In summary, the following criteria where chosen for model selection in this work:
\begin{enumerate}
    \item It is possible to get the full model weights.
    \item The selected models ought to be decently capable \glspl{causal}.
    \item Ceteris paribus, a smaller model is better.
\end{enumerate}



\subsection{OPT, BLOOM (Not Used)}
\paragraph{OPT}\label{par:opt}\label{sub:OPT}
The initial model set out for this work was \model{OPT} \cite{zhang_opt_2022}, a 175 billion parameter open-source \gls{LLM} trained by \gls{meta}, with partially similar capability as \gls{GPT3}. During early literature research, we encountered the similar but slightly more capable \model{BLOOM}.

\paragraph{BLOOM}\label{par:bloom}\label{sub:BLOOM}
\model{BLOOM} \cite{workshop_bloom_2022} is a 176 billion parameter open-source \gls{LLM} trained by a cooperation of numerous organizations, spearheaded by \gls{hf} and \gls{Google}. When compared to \model{OPT} across \gls{NLP} benchmarks, \model{BLOOM} appears to perform marginally better.

\paragraph{Reasons for Using Neither Model}
The original plan for this work would use \model{OPT} as the only model. During early literature research, it seemed that \model{BLOOM} would be slightly more capable, which changed the intention to compare both.
Not soon after, the smaller and seemingly much more capable \model{llama} was released, which prompted the decision of creating a model-agnostic pipeline instead, focusing on \model{llama} first.
See the next \subref{llama} for more details on \model{llama}.

\subsection{LLaMa (Used)}\label{sub:llama}
\model{llama} is a family of open-access \glspl{LLM} provided by \gls{meta} with sizes ranging from 7 billion to 65 billion parameters, and capabilities comparable to, and sometimes beating \gls{SOTA} (including the substantially larger \gls{GPT3}) at the time of release \cite{touvron_llama_2023}.
\model{llama} can be seen as the first culmination of progress on \glspl{LLM} up to this point, in one place.

\model{llama} is not instruction fine-tuned. See \subref{instruct} for more details on instruction fine-tuning.
For instruction fine-tuned variants of \model{llama}, see \subref{alpaca} on \model{alpaca} or \subref{vicuna} on \model{vicuna}.

\subsection{Alpaca (Not Used)}\label{sub:alpaca}
The \model{alpaca} Project \cite{tatsulab_2023} aims to build and share an instruction fine-tuned \model{llama} model.
Due to uncertainty with the \model{llama} licence which this model is based on (it was fully released a mere two weeks after \gls{llama} was first announced), no model weights where released officially.
Instead, all scripts and training data to fine-tune your own \model{alpaca} based on existing \model{llama} weights where provided.
Fine-tuning on a large dataset becomes impractical for larger model variants due to rapidly increasing resource requirements.
For this reason, it was decided against including \model{alpaca} in the benchmark.


\subsection{Vicuna (Used Partially)}\label{sub:vicuna}
\model{vicuna} is a family of instruction fine-tuned \model{llama}-variants, released by \gls{lmsys}. It is built on top of the training recipe of \model{alpaca}.
However, not all weights of the corresponding \model{llama} sizes are available.
The largest \model{llama}-model (65B) does not have a corresponding \model{vicuna} derivative available.
In a tournament format between different \glspl{LLM}, \model{vicuna} provided user-preferred answers more often than \model{llama} and \model{alpaca} \cite{zheng_judging_2023}, among others.
Thus, before the release of \model{falcon} and \model{llama2}, \model{vicuna} was generally seen as the most capable instruct-based model, which is why it was included in this benchmark.


\subsection{LLaMa 2 (Used)}\label{sub:llama2}
\gls{meta} released \model{llama2} \cite{touvron_llama2_2023} only a few months after the release of its predecessor.
They introduced few fundamental changes when compared to \model{llama}.
The main differences include making use of \gls{GQA} for the first time, and training on more tokens.
For each size, \model{llama2} was released in four versions: 1) the base model 2) a `helpful'-variant trained with human-feedback 3) a `chat' variant optimized for dialogue and 4) a combined `helpful' and `chat' variant.


\subsection{Falcon (Used)}\label{sub:falcon}
The \model{falcon} \cite{zxhang_falcon_2023} family of language models are created by the Abu Dhabi-based \gls{tii}.
Since its release, \model{falcon} is at the top of most benchmarks between open-access models (in each respective parameter size category) \cite{zxhang_falcon_2023}.
It appears to rival some of the most capable closed-access models such as \gls{PaLM} in capability.

The better performance of \model{falcon} for most tasks is assumed to mostly be the result of longer training and higher-quality data sets \cite{zxhang_falcon_2023}.

Recently, a new closed-access 180 billion parameter \model{falcon} variant was announced \cite{tii_falcon180b_2023}. \model{falcon}-180B is not included in this benchmark.


\subsection{GPT4 (Not Used)}\label{sub:GPT4}
\model{GPT4} is the fourth generation \gls{GPT} model from \gls{OpenAI} \cite{openai_gpt4_2023}.
It is the single most capable \acrlong{LM} we currently know of.
However, it is not open-source and only accessible through interfaces provided by \gls{OpenAI}.
Additionally, \gls{OpenAI} continues to work on, change, and sometimes degrade the capabalities of \model{GPT4} \cite{chen_how_2023}.
Even timestamp-versioned, 'unchanging' models have been claimed to measurably change in behaviour \cite{jw1224_hn}.
This makes it a hard target for comparison.

\model{GPT4} does not fulfill the criteria of being open-access, and is thus not compared in this work.


\subsection{Final List}\label{sub:list}
In conclusion, we used the following models and sizes of the aforementioned:
\begin{itemize}
    \item \model{llama} 7B, 13B, 30B, 65B (See \subref{llama} for more details on the model)
    \item \model{vicuna} 7B, 13B, 33B (See \subref{vicuna} for more details on the model)
    \item \model{llama2} 7B, 13B, 70B (See \subref{llama2} for more details on the model)
    \item \model{falcon} 7B, 40B (See \subref{falcon} for more details on the model)
    \item \model{falcon}-instruct 7B, 40B (See \subref{falcon} for more details on the model)
\end{itemize}


\section{Prompts Used}\label{sec:prompts}
\glspl{LLM} are capable of very generic tasks, based on the input they are asked to repsond to.
The way to get them to solve a task as requested is by \textit{prompting} the model with a certain input.
This is particularly emintent in instruct-based models (See \subref{instruct} for more details on instruction-based finetuning).

We have not put much effort in figuring out the best prompts, primarily because any amount of fine-tuning would be more effective than doing so. You can read more on what went wrong trying to do that in the following \secref{sft}.

As used \textit{guidance} \cite{guidance_2023}, and specifically the library \texttt{jsonformer} \cite{1rgs_2023}, for getting structured information as an output.
In effect, guidance provides `guard rails' for models generating output.
Specifically, the model does not have to generate the tokens for the structure of json, but only the tokens for the content of the json.

\code{schema.py}{schema}{The schema provided for the model to follow. Model output termination would happen after generation of a token for `\mintinline{python}{"}' for strings or `\texttt{,}' for numbers, or a number of other dedicated 'end of generation' tokens. See \coderef{output} for what an output for this schema might look like.}

You can see the schema we used for guidance in \coderef{schema}. Additionally, you can find the full prompt used in \coderef{prompt} and an example output in \coderef{output}.

\code{prompt.py}{prompt}{Prompt used to generate output. \mintinline{python}{"{output}"} delineates where the model provides an answer. See \coderef{output} for what may be filled in.}

\code{example_output.py}{output}{Exemplary output based on the prompt shown in \coderef{prompt}.}

\textbf{Note} that while all code includes \texttt{additive}, later figures in \chapref{results} do not.
This is due to mistaken modeling of the task on our part, which resulted in a horrendous accuracay. \todo{put in worst and best accuracy scores for additives}

% \subsection{Prompt Engineering}\label{sub:engineering}
% Answers, even to the same prompts, across models and even from the same model, can vary substantially \cite{chen_how_2023}.
% Thus, a short-lived 'discipline', Prompt Engineering, emerged.
% Prompt Engineering attempted to find out how to write prompts to get the best results, out of either specific or all models.
% It was quickly found out that this is a task that can be automated with the help of a \gls{LLM} \cite{zhou_large_2022}.

% additional relevancy for applications where potentially hostile users can directly or indirectly prompt a model, and thus 'Prompt Injection Attacks' where born \cite{greshake_more_2023}.

% \subsection{Prompt Guidelines}\label{sub:guidelines}
% \todo{totally rewrite}
% A few general guidelines for prompts empirically emerged (mostly through people sharing results on twitter):
% \begin{itemize}
%     \item Guidance for everything structure-based \cite{guidance_2023}
%     \item Chain-Of-Thought for reasoning \cite{wei_chainofthought_2022}
%     \item Reflexion for even bigger models \cite{shinn_reflexion_2023}
% \end{itemize}

% \subsection{Prompts Used}\label{sub:prompts}


\section{Data Source}\label{sec:data}
We used 905 synthesis paragraphs which were used to create parts of the publicly accessible labels in the databases SynMOF\_A and SynMOF\_M \cite{luo_mof_2022}.
We defaulted to get a label from SynMOF\_M (manually annotated) if it was available, and manually confirmed the validity of the SynMOF\_A label (generated from automatic extraction) if it existed.
\todo{check again if default was A or M}
In total, we had labels for 905 synthesis paragraphs which we fully utilisied.
All proportional results in the later \chapref{results} are based on the accuracy over 905 items.

\section{Supervised Fine Tuning}\label{sec:sft}
\glspl{LLM} tend to have been pretrained on so much data that they have become generally capable of most tasks even with no fine-tuning \cite{brown_language_2020}.
Fine-tuning a pretrained model instead of a non-pretrained model requires substantially less compute, and sometimes more importantly, hardly any examples to train on to achieve similar results for most tasks \cite{gaddipati_comparative_2020}.

See \secref{training} for more details on pre-training a \gls{LLM} and \subref{finetune} for details on finetunig.

In the end, we abandoned fine-tuning due to time constraints. See \secref{con:sft} for conclusions and \secref{out-sft} for an outlook on fine-tuning.

What follows are two excerpts of what we encountered when attempting to fine-tune \glspl{LLM} using \gls{transformers}.
While they are by no means exhaustive, other failed approaches followed a similar pattern of wrong or lacking documentation among numerous locally fixed bugs.

\subsection{Excerpt 1: Broken Models}\label{sub:brokenft}
Since we have a custom dataset, we also need to write a custom dataloader.
An item of the dataloader can just be the text of the synthesis paragraph -- in a dictionary, as a very specific key.
There is however, only partial and conflicting documentation on the existance and usage of these keys.
\todo{continue writing excerpt 1 from here on}

nuances: tokenization of dataset prior to training. however, which part is doing what?

building custom dataset: array with dicts, with the three required keys \verb`input_ids`, \verb`attention_maska`, and \verb`labels`. curiously, neither is documented particularly well so we tried what is recommended in various tutorials and official sources (e.g. microsoft \cite{deepspeedexamples_2023}): 

put the \verb`token_ids` received from tokenization to both \verb`input_ids` and \verb`labels`.

This did result in a model with differing weights than it had before. This model however, was broken as it did not generate anything that was not an EOS-token.
this token is usually used as a stopping criterion during generation.
? resulted in broken model, probably learned that it's 'finished', only outputting EOS tokens. Not sure if doing this otherwise would actually change anything though

Attempts at mask manipulation: not possible with causalLMs (they are all of this type)

\subsection{Excerpt 2: Broken Libraries}\label{sub:libraries}
In a later attempt wie tried using the high-level \gls{hf} \verb`trl` (Transformer Reinforcement Learning) library, which seems to be built for our use-case exactly.

However, this library is at best research-grade. The examples, while working with only a few lines, obscure the inner workings of the library.
And good luck: it's also not documented. There is the \verb`DataCollatorForCompletionOnlyLM` collator, which takes a tokenizer, but also doesn't tokenize?!?
\todo{rewrite subsection on broken libraries}

examples only have \verb`text` field, there is a formatting function and whatnot, but this implies tokenization is happening later. nope, errors with 'missing field \verb`token_ids`'.

trying out various things didn't work, until we ultimately didn't have time to continue.

SFT: a lot of magic that isn't documented properly, at all. Couldn't get it to run, gave up due to time limit.

% \mintinline{python}{trl} library \cite{hf_trl_supervised}

\section{Criteria for Equality}\label{sec:equality}
In this section we try to list the criteria we used to define equality between a result from a \gls{LLM} and the target label.
\todo{write section on criteria for equality}

for temperature and time we did conversion between units (not super straightforward), models had a bit of unit confusion
(sometimes adding too many or too few zeros, though also often getting it right)

solvents and additives: getting cid and comparing it (if it can be gotten in the first place though)

