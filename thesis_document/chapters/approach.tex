\chapter{Approach}\label{chap:approach}
This chapter describes all relevant decisions of the approach taken, as well as their reasons.

First, \secref{impl} broadly describes the implementation, as well as libraries and frameworks used.
In \secref{models} descriptions and references to the various models considered for this work can be found.
\secref{prompts} discusses how the models where prompted in detail, and why little effort and experimentation was done with different prompts.
The dataset used, and how it was processed is described in \secref{data}, before it is depicted how the comparison for equality was done in \secref{equality}.


\section{Implementation}\label{sec:impl}
All source code for this work can be found at \url{https://github.com/fkarg/mthesis}, and a tag will mark the state at the time of submission.
\todo{publish on day of submission}

It became appearent during literature research that a comparison between different models would be valuable, and that additional models can be expected to be released in time to be included in this work.
Thus, almost all code ought to be model-agnostic.
The \acrlong{transformers} library was chosen due to being a well-established framework providing abstractions to load, manipulate and train any deep learning architecture in a standardized format.
Additionally, all open-access \glspl{LLM} are available directly through the \gls{hf} portal.

\paragraph{Dependencies}
Most other dependencies used are either straightforward (\texttt{torch}, \texttt{einops}, \texttt{accelerate}, \texttt{bitsandbytes} to get the models to run) or common ecosystem choices (e.g. \texttt{typer} and \texttt{rich} for the cli interface; \texttt{pubchempy} to resolve and convert chemical compounds; etc).

\paragraph{Modularity}
Proportionally speaking, the main module (with 36\%), dataloader (with 23\%), and unit conversion module (with 11\%) have the highest \gls{LOC} counts. Everything else is split in six more supporting modules.


\section{Language Models Considered}\label{sec:models}
The following sections describe the various models considered for the benchmark, and the reasons for or against their inclusion.
Before that, \subref{criteria} depicts simple criteria for model selection.

See \secref{basics} for an overview and broad categorization of the models mentioned here.


\subsection{Criteria}\label{sub:criteria}
First, the model weights need to be available.
This is necessary to run the model in a self-hosted manner.
Practically, this is a constraint towards open-source or at least open-access models (their license needs to allow for academic research).
% \todo{reason for criteria: people want to fine-tune models on their own potentially sensitive data}

Second, at this point dozens of \glspl{LLM} have been trained, and weights made available.
One model was trained purely as a marketing pitch to sell computing systems \cite{dey_cerebrasgpt_2023}.
To vastly condense the number of models to consider, each base model ought to have demonstrated fundamental capability in other domains.
Additionally, due to the nature of the task, it is valuable to include both base models and instruct-based derivatives for comparison.

Third, all else being equal, a smaller model is preferred.
This is due to smaller models being less resource intensive to evaluate and fine-tune.
Smaller variants of larger models additionally enable faster and less resource intensive iteration during development.

In summary, the following criteria where chosen for model selection in this work:
\begin{enumerate}
    \item It is possible to get the full model weights.
    \item The selected models ought to be decently capable \glspl{causal}.
    \item Ceteris paribus, a smaller model is better.
\end{enumerate}



\subsection{OPT, BLOOM (Not Used)}
\paragraph{OPT}\label{par:opt}\label{sub:OPT}
The initial model set out for this work was \model{OPT} \cite{zhang_opt_2022}, a 175 billion parameter open-source \gls{LLM} trained by \gls{meta}, with partially similar capability as \gls{GPT3}. During early literature research, we encountered the similar but slightly more capable \model{BLOOM}.

\paragraph{BLOOM}\label{par:bloom}\label{sub:BLOOM}
\model{BLOOM} \cite{workshop_bloom_2022} is a 176 billion parameter open-source \gls{LLM} trained by a cooperation of numerous organizations, spearheaded by \gls{hf} and \gls{Google}. When compared to \model{OPT} across \gls{NLP} benchmarks, \model{BLOOM} appears to perform marginally better.

\paragraph{Reasons for Using Neither Model}
The original plan for this work would use \model{OPT} as the only model. During early literature research, it seemed that \model{BLOOM} would be slightly more capable, which changed the intention to compare both.
Not soon after, the smaller and seemingly much more capable \model{llama} was released, which prompted the decision of creating a model-agnostic pipeline instead, focusing on \model{llama} first.
See the next \subref{llama} for more details on \model{llama}.

\subsection{LLaMa (Used)}\label{sub:llama}
\model{llama} is a family of open-access \glspl{LLM} provided by \gls{meta} with sizes ranging from 7 billion to 65 billion parameters, and capabilities comparable to, and sometimes beating \gls{SOTA} (including the substantially larger \gls{GPT3}) at the time of release \cite{touvron_llama_2023}.
\model{llama} can be seen as the first culmination of progress on \glspl{LLM} up to this point, in one place.

\model{llama} is not instruction fine-tuned. See \subref{instruct} for more details on instruction fine-tuning.
For instruction fine-tuned variants of \model{llama}, see \subref{alpaca} on \model{alpaca} or \subref{vicuna} on \model{vicuna}.

\subsection{Alpaca (Not Used)}\label{sub:alpaca}
The \model{alpaca} Project \cite{tatsulab_2023} aims to build and share an instruction fine-tuned \model{llama} model.
Due to uncertainty with the \model{llama} licence which this model is based on (it was fully released a mere two weeks after \gls{llama} was first announced), no model weights where released officially.
Instead, all scripts and training data to fine-tune your own \model{alpaca} based on existing \model{llama} weights where provided.
Fine-tuning on a large dataset becomes impractical for larger model variants due to rapidly increasing resource requirements.
For this reason, it was decided against including \model{alpaca} in the benchmark.


\subsection{Vicuna (Used Partially)}\label{sub:vicuna}
\model{vicuna} is a family of instruction fine-tuned \model{llama}-variants, released by \gls{lmsys}. It is built on top of the training recipe of \model{alpaca}.
However, not all weights of the corresponding \model{llama} sizes are available.
The largest \model{llama}-model (65B) does not have a corresponding \model{vicuna} derivative available.
In a tournament format between different \glspl{LLM}, \model{vicuna} provided user-preferred answers more often than \model{llama} and \model{alpaca} \cite{zheng_judging_2023}, among others.
Thus, before the release of \model{falcon} and \model{llama2}, \model{vicuna} was generally seen as the most capable instruct-based model, which is why it was included in this benchmark.


\subsection{LLaMa 2 (Used)}\label{sub:llama2}
\gls{meta} released \model{llama2} \cite{touvron_llama2_2023} only a few months after the release of its predecessor.
They introduced few fundamental changes when compared to \model{llama}.
The main differences include making use of \gls{GQA} for the first time, and training on more tokens.
For each size, \model{llama2} was released in four versions: 1) the base model 2) a `helpful'-variant trained with human-feedback 3) a `chat' variant optimized for dialogue and 4) a combined `helpful' and `chat' variant.


\subsection{Falcon (Used)}\label{sub:falcon}
The \model{falcon} \cite{zxhang_falcon_2023} family of language models are created by the Abu Dhabi-based \gls{tii}.
Since its release, \model{falcon} is at the top of most benchmarks between open-access models (in each respective parameter size category) \cite{zxhang_falcon_2023}.
It appears to rival some of the most capable closed-access models such as \gls{PaLM} in capability.

The better performance of \model{falcon} for most tasks is assumed to mostly be the result of longer training and higher-quality data sets \cite{zxhang_falcon_2023}.

Recently, a new closed-access 180 billion parameter \model{falcon} variant was announced \cite{tii_falcon180b_2023}. \model{falcon}-180B is not included in this benchmark.


\subsection{GPT4 (Not Used)}\label{sub:GPT4}
\model{GPT4} is the fourth generation \gls{GPT} model from \gls{OpenAI} \cite{openai_gpt4_2023}.
It is the single most capable \acrlong{LM} we currently know of.
However, it is not open-source and only accessible through interfaces provided by \gls{OpenAI}.
Additionally, \gls{OpenAI} continues to work on, change, and sometimes degrade the capabalities of \model{GPT4} \cite{chen_how_2023}.
Even timestamp-versioned, 'unchanging' models have been claimed to measurably change in behaviour \cite{jw1224_hn}.
This makes it a hard target for comparison.

\model{GPT4} does not fulfill the criteria of being open-access, and is thus not compared in this work.


\subsection{Final List}\label{sub:list}
In conclusion, we used the following models and sizes of the aforementioned:
\begin{itemize}
    \item \model{llama} 7B, 13B, 30B, 65B (See \subref{llama} for more details on the model)
    \item \model{vicuna} 7B, 13B, 33B (See \subref{vicuna} for more details on the model)
    \item \model{llama2} 7B, 13B, 70B (See \subref{llama2} for more details on the model)
    \item \model{falcon} 7B, 40B (See \subref{falcon} for more details on the model)
    \item \model{falcon}-instruct 7B, 40B (See \subref{falcon} for more details on the model)
\end{itemize}



\section{Prompts}\label{sec:prompts}
\glspl{LLM} are capable of very generic tasks, based on the input they are asked to respond to.
\textit{Prompting} a model with a certain input gives more fine-grained control over the output and can provide additional structure, information, and suggestions for solving a specified task.
This is particularly eminent in models fine-tuned for instruction-based or chat-based interaction.

In this work, the main effort was put towards creating a pipeline for fine-tuning all models for the specified \gls{NER} task.
This was due to the expectation that even small amounts of fine-tuning would be more effective in guiding model outputs than the best prompt could be.
% Thus, prompting was given only cursory attention.
Results of fine-tuning attempts from this work can be found in the later \secref{res:sft}.

\paragraph{Structured Output}
\texttt{guidance} \cite{guidance_2023}, originally from \gls{microsoft} research, allows more effective control over \glspl{LLM} than traditional prompting or chaining does.
In effect, \texttt{guidance} is a harness around a \gls{LLM}, providing support for the model in generating structured information, and making use of additional output structures such as CoT \cite{wei_chainofthought_2022} to result in higher-quality outputs.

For this work, specifically the library \texttt{jsonformer} \cite{1rgs_2023} is used, which does not provide the full feature suite of the \texttt{guidance} library.
At the time of deciding, \texttt{guidance} did not support models through the \gls{transformers} library yet.

The schema used for guidance can be seen in \coderef{schema}.
Additionally, the full prompt used can be seen in \coderef{prompt}, and an example output in \coderef{output}.


\code{schema.py}{schema}{The schema provided for the model to follow. Model output termination would happen after generation of a token for `\mintinline{python}{"}' for strings or `\texttt{,}' for numbers, or a number of other dedicated 'end of generation' tokens, e.g. \texttt{<EOS>}. See \coderef{output} for what an output for this schema might look like.}

\code{prompt.py}{prompt}{Prompt used to generate output. \mintinline{python}{"{output}"} delineates where the model provides an answer. See \coderef{output} for what may be filled in.}

\code{example_output.py}{output}{Exemplary output based on the prompt shown in \coderef{prompt}, and schema shown in \coderef{schema}.}


% \subsection{Prompt Engineering}\label{sub:engineering}
% Answers, even to the same prompts, across models and even from the same model, can vary substantially \cite{chen_how_2023}.
% Thus, a short-lived 'discipline', Prompt Engineering, emerged.
% Prompt Engineering attempted to find out how to write prompts to get the best results, out of either specific or all models.
% It was quickly found out that this is a task that can be automated with the help of a \gls{LLM} \cite{zhou_large_2022}.

% additional relevancy for applications where potentially hostile users can directly or indirectly prompt a model, and thus 'Prompt Injection Attacks' where born \cite{greshake_more_2023}.

% \subsection{Prompt Guidelines}\label{sub:guidelines}
% \todo{totally rewrite}
% A few general guidelines for prompts empirically emerged (mostly through people sharing results on twitter):
% \begin{itemize}
%     \item Guidance for everything structure-based \cite{guidance_2023}
%     \item Chain-Of-Thought for reasoning \cite{wei_chainofthought_2022}
%     \item Reflexion for even bigger models \cite{shinn_reflexion_2023}
% \end{itemize}

% \subsection{Prompts Used}\label{sub:prompts}


\section{Data Source}\label{sec:data}
As data source, 778 synthesis paragraphs and their corresponding labels from the publicly accessible database SynMOF\_M \cite{luo_mof_2022} where used.
Each synthesis paragraph describes the creation procedure of a \gls{MOF}.
This work focused on the basic parameters \ttemp, \ttime and \tsolv, though additional parameters could be added quickly to the schema discussed in the prior \secref{prompts}.

% TODO: maybe describe 'every synthesis has one
% or multiple temp, duration, solvent,
% optionally additive, and other parameters???

The labels from the SynMOF\_M database where manually annotated in prior work \cite{luo_mof_2022}.
% In total, those 905 labels and synthesis paragraphs where fully utilized during evaluation.
All proportional results in the later \chapref{results} are based on the accuracy of over all 778 items.
For training purposes, this dataset would have been split in dedicated datasets for test and training.
% A split in test and training dataset was performed for training, but this should not influence the accuracy for evaluation.
% Only a fraction of that would have been used for training purposes.
%, as fine-tuning with a few hundred examples is more than sufficient \cite{dunn_structured_2022}.

\section{Criteria for Equality}\label{sec:equality}
This section defines the criteria for determining equality between a result from a \gls{LLM} and the target label.
The criteria for \ttemp and \ttime and dealing with unit conversions is described in \subref{ttunit}.
Then, \subref{compsolv} detailes how compounds are compared.

\subsection{Time and Temperature}\label{sub:ttunit}
In the dataset (See \secref{data} for more details on the data source), all temperature information is encoded in degrees celsius, and all time information in hours.
Without a field for the unit (See \secref{prompts} for the prompts and structure used), models would use arbitrary units, often those used in the paragraph they are extracting from.
Since the task is not accurate unit conversion, but information extraction, a field for \ttemp and \ttime units was added.

Unit conversions for \ttemp and \ttime happen automatically before comparison and convert to a unified format, degrees celsius and hours respectively.
This ensures that durations of both '24h' and '1 day' are seen as equal, even though the strings are different.

\subsection{Compounds}\label{sub:compsolv}
Instead of names of chemical compounds, the database (See \secref{data} for more information on the data source) contains the \texttt{pubchempy}-\texttt{cid} (compound id) as the labels for solvents and additives (if applicable).
Most compounds have multiple different synonymous names they are known by, e.g. `water' has one \cid (which is 962) and a list of 319 distinct strings it can get resolved from.
Surprisingly, while the list of synonyms includes both `distilled water' and `H2O', it does not include `distilled H2O', which is mentioned verbatim in eight of the 778 synthesis paragraphs, and suggested as an answer by some models.
For more details on compound resolution problems, see \subref{solv}.

For each answer provided from the model for \tadd and \tsolv, an attempt at resolving the \cid is made.
If a \cid is found, it is compared with the label \cid.
An answer is counted as `wrong' when the resolved \cid is different to the label, but also when resolving fails.
Thus, the answer `distilled H2O' would be counted as `wrong', since it could not be resolved to a valid \cid.

% \subsection{F-Score}\label{sub:fscore}
% \draft{
% we don't attempt to assign categories to sections, instead extract the information directly
% }
% \todo{explain why fscore doesn't make much sense for us}

% For training purposes, the custom dataloader would search for any of the synonyms in the paragraph and use it as 'label'-text if found, or the first synonym if none could be easily identified.

