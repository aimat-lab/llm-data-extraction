\chapter{Introduction}\label{chap:introduction}
\todo{find a first section for introduction chapter}

% clear and concise:
% Presentation of the topic and the background
% Motivation of importance
% Research gap, research question
% Structure of the thesis
% Short summary of main results


\glspl{LLM} have recently become a hot topic of public attention, which started with the release of \gls{ChatGPT}.
% notoriety


\gls{ML} models are increasingly used in screening steps for materials discovery and property prediction \cite{saal_machine_2020, luo_mof_2022, choudhary_recent_2022}.
But the amount of data accessible for such models is limited, particularly for new, experimental or very recent work.
\todo{write introduction chapter}

\draft{
It has been well established that more high-quality data makes models more accurate \cite{hoffmann_empirical_2022}.
}

\draft{
Trying to use \glspl{LLM} for \gls{NER} on scientific text synthesizing \glspl{MOF}.
}

\draft{
basic problem: access to knowledge buried deep within unknown papers.
A vast amount of hard-earned scientific knowledge is scattered across millions of research literature.
Often, this research is not in standardized machine-readable formats, which makes it difficult or impossible to
build on a large part of prior work using powerful tools to extract further knowledge.
}

% \section{Basics}

% \section{Motivation}
% Take for example the field of synthesizing Metal-Organic Frameworks (MOFs)
% \cite{zhou_introduction_2012}. While numerous detailed descriptions of
% synthesis procedures exist, they are not available in machine-readable formats,
% which prevents effective application of state-of-the-art techniques such as
% automated experimentation \cite{shi_automated_2021} or synthesis prediction
% \cite{luo_mof_2022}. Thus, we intend to create a pipeline for deriving
% machine-readable information on MOF synthesis parameters from given questions
% on provided scientific articles.

% Motivation
% \begin{itemize}
%     \item vast amount of material science knowledge scattered across papers
%     \item often non-machine readable formats
%     \item makes it difficult or impossible to build on (all) prior work
%     \item results in tremendous duplicates and other unnecessary work
%     \item a lot of insights to be gained from such a database, as well as automated experimental design (and possibly execution)
% \end{itemize}

% How to construct a Nature summary paragraph
% One or two sentences providing a basic
% introduction to the field, comprehensible to
% a scientist in any discipline.

%  Two to three sentences of more detailed
%  background, comprehensible to scientists in
%  related disciplines.

%  One sentence clearly stating the general
%  problem being addressed by this particular
%  study.

%  One sentence summarizing the main result
%  (with the words “here we show” or their
%  equivalent).

%  Two or three sentences explaining what the
%  main result reveals in direct comparison to
%  what was thought to be the case previously,
%  or how the main result adds to previous
%  knowledge.

%  One or two sentences to put the results
%  into a more general context.

%  Two or three sentences to provide a broader
%  perspective, readily comprehensible to a
%  scientist in any discipline, may be
%  included in the first paragraph if the
%  editor considers that the accessibility of
%  the paper is significantly enhanced by
%  their inclusion. Under these circumstances,
%  the length of the paragraph can be up to
%  300 words. (This example is 190 words
%  without the final section, and 250 words
%  with it).


\section{Scientific Question}\label{sec:question}

The goals of this work are threefold:
\begin{enumerate}
    \item Demonstrate zero-shot automated information extraction from scientific literature using open-access \glspl{LLM}.
    \item Benchmark and compare the accuracy of currently available open-access \glspl{LLM} for the automated information extraction from scientific literature.
    \item Attempt fine-tuning of open-access \glspl{LLM} in order to increase accuracy.
\end{enumerate}

As part of this work, a highly flexible automated pipeline for the extraction of non-machine readable information in \gls{MOF} synthesis will be created.
The approach chosen in this work will be discussed in more detail \chapref{approach}, where the implementation is described in \secref{impl}, the models used in \secref{models}, and \secref{data} describes the source of data.


\newpage
collection of todos
\todo{maybe move glossary before introduction}
\todo{Do I want to introduce e.g. context length properly?}
\todo{Make sure I use passive voice everywhere (no 'we' or 'our')}
\todo{mention different modeling of NER task ... Results?}
